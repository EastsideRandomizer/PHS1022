\documentclass[12pt]{article}
\usepackage{amsmath}
\begin{document}

\title{PHS1022 - Potential Energy of Parallel Plate Capactiors}
\date{22 August, 2012}
\author{Gordon Ng}
\maketitle
\pagebreak
\tableofcontents
\pagebreak
\section{Potential Energy}
\begin{align}
U_{C}=\frac{1}{2}U (\Delta V)^2
\end{align}
\section{Dielectrics}
Dielectrics which is a insulator inserted between the electrodes of capacitor increases the capacitance of capacitor.

When the insulator is polarised, it leads to excess charges building up in both side of electrodes.

The dielectic can be said as two sheet of electric surface. It induces a $E_{induced}$ in a direction which is opposite to the direction of a empty parallel plate capacitor.  But the magnitude of $E_{induced}$ is lower than $E_0$ hence it does not totally cancels put the electric field. Hence the sum of electric field can be represented as:
\begin{align}
\vec{E}=\vec{E_0}+\vec{E_{induced}}
\end{align}

\begin{align}
\Delta V_c &= Ed = \frac{E_0}{K}d = \frac{\Delta V_O}{K} \\
C&=\frac{Q}{\Delta V}=K C_0
\end{align}

\subsection{Dielectric Strength}
Dielectrics strength is measure as Millions Volt per metre. $(10^6V m^{-1})$
\pagebreak
\section{RC Circuits}
It consists of Resistor and Capacitor (and indeed a Voltage Source.)

It can be analyzed by Kirchhoff's Loop Law:
\begin{align}
\Delta V_C + \Delta V_R &= 0 \\
\frac{Q}{C}-IR &=0 \\
I&=\frac{dq}{dt}=-\frac{dQ}{dt}\\
\end{align}
Hence:
\begin{align}
Q(t)&=Q_0e^{-\frac{t}{\tau}} \\
\text{Where $\tau$:}\\
\tau &= RC
\end{align}
Hence current:
\begin{align}
I(t)&=\frac{d(Q_0e^{-\frac{t}{\tau}})}{dt} \\
I(t)&=\frac{Q_0}{RC}e^{-\frac{t}{\tau}} \\
I_0&=\frac{Q_0}{RC}
\end{align}

Charge:
\begin{align}
Q(t)&=Q_{max}(1-e^{-\frac{t}{\tau}})  \\
Q_{max}&=C \Delta V_{max}
\end{align}
\end{document}