\documentclass[12pt]{article}
\begin{document}

\title{Electric Flux and Gauss' Law - Continued}
\date{10 August, 2012}
\author{Gordon Ng}
\maketitle
\pagebreak
\tableofcontents
\pagebreak
\begin{itemize}
\section{Examples}
\subsection{Sphere of Charge}
\item Treat is as a point charge.
\subsubsection{Outside the Sphere}
\begin{displaymath}
\Phi_e=\int{\vec{E}}d\vec{A}=\int EdA
\end{displaymath}
\subsubsection{Inside of sphere}
\begin{displaymath}
\Phi_e=\int{\vec{E}}d\vec{A}=\int EdA
\end{displaymath}
\begin{displaymath}
\vec{E}=\frac{KQr}{R^3}r
\end{displaymath}
\item r is the radius of Gaussian surface inside surface of charge.
\item the field inside the sphere increas linearly with distance

\subsection{Line Charge}
\item We choose a length L.
\item there is no electric flux along the end point due to $d\vec{A}$ perpendicular with $\vec{E}$.

\begin{displaymath}
\Phi_e=\int{\vec{E}}d\vec{A}
\end{displaymath}
\begin{displaymath}
\lambda = \frac{Q_{encl}}{L}
\end{displaymath}
\item L should not appear on the answer
\begin{displaymath}
\int{E}dA = \frac{lambda L }{\varepsilon_0}
\end{displaymath}
\begin{displaymath}
E = 2 \pi
\end{displaymath}
\end{itemize}
\section{Parallel-plate capacitor}
$|\vec{E}|$ = 2 times the field strength for single $ \infty$ plane $= \frac {\eta}{\varepsilon_0}$
\end{document}