\documentclass[12pt]{article}
\usepackage{amsmath}
\usepackage{tikz}
\numberwithin{equation}{section} 
\begin{document}
\title{Change in Voltage}
\date{17 August, 2012}
\author{Gordon Ng}
\maketitle
\pagebreak
\tableofcontents
\pagebreak
\section{Introduction to Magnetism}
\begin{itemize}
\item The Magnetic field is similar to a electric dipole field.
\end{itemize}

\section{Notation}
\begin{itemize}
\item Into the page : $\times$
\item Out of the page : $\bullet$
\end{itemize}
\section{Field Strength}
\begin{align}
B\propto \frac{1}{r^2}
\end{align}
\section{Biot-Savart Law}
\begin{align}
B_{PC}=\frac{\mu_o}{4 \pi} \frac{qv sin \theta}{r^2}
\end{align}
Where
\begin{align*}
\mu_0 &= 4 \pi \times 10^{-7} TmA \\
q &= \text{Charge (C)} \\
v &= \text{Velocity of Charge} \\
\theta &= \text{Angle between velocity vector and position vector} 
\end{align*}

\subsection{Magnetic Field For A Long Current Carrying wire}
\begin{align}
B_{PC}=\frac{\mu_o}{2 \pi}\frac{I}{d}
\end{align}

\subsection{Magnetic Field along a current loop axis}
\begin{align}
B_{PC}=\frac{\mu_o}{2}\frac{IR^2}{(z^2+R^2)^{\frac{3}{2}}}
\end{align}
\subsubsection{If z= 0}
\begin{align}
B_{PC}&=\frac{\mu_o}{2}\frac{I}{R^{\frac{3}{2}}} \\
\intertext{If There is N Rings:}
B_{PC}&=\frac{\mu_o}{2}\frac{NIR^2}{(z^2+R^2)^{\frac{3}{2}}}
\end{align}
\section{Symmetry And Ampere's Law}
They symmetry between Gauss's Law and 
\\
Gauss's law:
\begin{align}
\Phi_{e} *= \oint E dA = \frac{Q_{in}}{\varepsilon_)}\\
\intertext{Line Integral}
\oint T ds &= n_1 ds_1 + n_2 ds_{1} \\
\intertext{Ampere's Law}
\intertext{When Magnetic Field is perpendicular to the Line}
\vec{B} \cdot d \vec{s}&=0 \\
\intertext{When the Magnetic Field is parallel to the Line}
\oint \vec{B} \cdot d \vec{s}&= Bl = B(2 \pi d)
\intertext{If we combine it with Biot-Savart Law:}
\oint \vec{B} \cdot d \vec{s}&= \mu_0 I
\end{align}
\subsection{Ampere's Law Inside a Wire}
In here we assume constant current denssity.
\begin{align}
  \intertext{Ratio of Area:}
  \frac{\pi r^2}{\pi R^2} &= \frac{r^2}{R^2} \\
  \intertext{Hence:}
  \oint \vec{B} \cdot d \vec{s}&= \frac{\mu_0 r^2 I}{R^2} \\
  \intertext{Replace Line Integral}
  2 \pi r B &= \frac{\mu_0 r^2 I}{R^2}\\
  \intertext{Rearrangement results in:}
  B&= \frac{\mu_0 I}{2pi R^2} r
\end{align}
\section{Solenoid}
The Solenoid should have a diameter which is fair smaller than the length of the coil.
They have such properties:
\begin{itemize}
\item The magnetic field inside a solenoid is uniform.
\item Infinitely Long
\end{itemize}
\subsection{Ampere's Law}
\begin{align}
  \intertext{From Biot-Savart Law:}
  \frac{\pi r^2}{\pi R^2} &= \frac{r^2}{R^2} \\
\oint \vec{B} \cdot d \vec{s}&= \mu_0 I
\end{align}
\subsection{Ampere's Law Inside a Wire}
In here we assume constant current denssity.
\begin{align}
  \intertext{Ratio of Area:}
  \frac{\pi r^2}{\pi R^2} &= \frac{r^2}{R^2} \\
  \intertext{Hence:}
  \oint \vec{B} \cdot d \vec{s}&= \frac{\mu_0 r^2 I}{R^2} \\
  \intertext{Replace Line Integral}
  2 \pi r B &= \frac{\mu_0 r^2 I}{R^2}\\
  \intertext{Rearrangement results in:}
  B&= \frac{\mu_0 I}{2pi R^2} r
\end{align}
\section{Solenoid}
The Solenoid should have a diameter which is fair smaller than the length of the coil.
They have such properties:
\begin{itemize}
\item The magnetic field inside a solenoid is uniform.
\item Infinitely Long
\end{itemize}
\pagebreak
\subsection{Ampere's Law}
\begin{align}
  \intertext{From Biot-Savart Law:}
 \oint \vec{B} \cdot d \vec{s}&= \mu_0 I\\
\intertext{Replace with line Integral:}
Bl&=\mu_0 N I\\
\intertext{Hence:}
B&= \frac{\mu_0 N I}{l} = mu_0 n I
\end{align}
n is the number of turns per unit length
 
\subsection{Doughnut Shape}
\begin{equation}
B= \frac{\mu_0 N I}{2 \pi r }
\end{equation}

\section{Frequency of a Cyclotron}
\begin{align}
f = \frac{qB}{a\pi m}
\end{align}

\section{Hall Effect}
\begin{align}
\Delta V_H = \frac{IB}{tne}
\end{align}
n= charge density.
e= charge of an Electron.

\section{Force on Wire Due To Magnetic Field}
\begin{align*}
F=BIL
\frac{\mu_o I_1 I_2}{2\pi d} 
\end{align*}

\section{Magnetic Moment}
\begin{align}
IA = Il2 = \mu
\end{align}
\section{Torque due to Magnetic Field}
\begin{align}
\vec{\tau} \times \vec{B} = \mu B sin \theta
\end{align}

\section{Electromagnetic Induction}

\section{Eddy Currents}
An current loop with a hole present acts like a capacitor with charges on either end of the hole. There IS Indeed an EMF even though no current will flow. 
 
\section{Faraday's Law}
\begin{equation}
\varepsilon=\left|\frac{d\Phi_{b}}{dt}\right|=\left|\vec{B} \cdot \frac{d\vec{A}}{dt}+\vec{A} \cdot \frac{d\vec{B}}{dt} \right|
\end{equation}
The direction of current is defined by Lenz's Law.
\section{Induced Electric Field}
If we divide the equation above into two parts. We can see that we have induced Electric-field which is non-Couloumbic. \\
This is generated by a \textbf{changing} magnetic field. Such force will still exert the force $F=qe$ on an charge. (Same as Coulombic Electric Field)

We know that :
\begin{align}
W=q \oint\vec{E}\cdot d \vec{s} \\
\intertext{Hence we can write that:}
\varepsilon= \oint \vec{E}\cdot d \vec{s} \\
\intertext{We can also conclude that:}
\oint \vec{E}\cdot d \vec{s} = A \left|\frac{d\vec{B}}{dt} \right| \\
\end{align}

If we have the pendulum we will have a Lorentz current which causes a Force.

\section{Modified Faraday's Law}
\begin{align}
\oint \vec{E}\cdot d \vec{s} =  - \frac{d\vec{\Phi_B}}{dt}
\end{align}
Where $ds = 2 \pi r$ if the line is a circle.

\section{Application of Faraday's Law}
\subsection{AC generator}
\subsubsection{EMF}
\begin{align}
\varepsilon =  - N \frac{d\vec{\Phi_B}}{dt} =  - N \frac{d\vec{BA cos (\omega t)}}{dt} = NBA sin (\omega t)
\end{align}
\pagebreak
\subsection{Transformer}
It does not work with DC (DC current only causes a constant Magnetic Field.
\begin{align}
\intertext{Ratio:}
\frac{V_1}{V_2}={N_1}{N_2} \\
\intertext{Current (Energy must be conserved):}
V_1I_1=V_2I_2\\
\frac{V_1}{V_2}=\frac{I_2}{I_1}\\
\frac{N_1}{N_2}=\frac{I_2}{I_1}
\end{align}

\subsection{Metal Detector}
The Transmitter coil causes a current in the receiver coil, but the metal can reduce the current in the receiver coil due to eddy currents.

\subsection{Inductor}
\begin{align}
\intertext{EMF:}
V=-L\frac{dI}{dt}\\
\intertext{Inductance}
L=\frac{\mu_0 N^2 A}{l}
\end{align}
The negative signs indicates that it opposes the change in current.

\subsection{Maxwell's Equations}
\begin{align}
\intertext{Gauss's Law for Magnetism}
\oint \vec{B} \cdot d \vec{A} =0  \\
\intertext{Ampere-Maxwell's Law}
\oint \vec{B} \cdot d\vec{s} = \mu_0 I_{through} + \epsilon_0 \mu_o \frac{d \Phi_e}{dt} \\ 
\intertext{Lorentz's Force Law}
F= q \vec{E} + q \vec{v} \times \vec{B} 
\end{align}

\section{Electromagnetic Waves}
An oscillating charge will create a changing magnetic and electric field which generates an EM Wave.
\begin{align}
\intertext{Poynting vector:}
|\vec{S}|=\frac{1}{\mu_0}\vec{E}\times\vec{B}=\frac{EB}{\mu_0}
\end{align}
Hence
\begin{align}
\frac{1}{2}\varepsilon_0 E^2 = \frac{1}{2 \mu_0} B^2
\end{align}

\end{document}