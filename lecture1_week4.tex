\documentclass[12pt]{article}
\usepackage{amsmath}
\begin{document}
\section{Potentials energy of point charges}
\begin{align*}
\delta U_{electric} &=\int_{R}^{\inf} F_{Elec}dr \\
&=\int_{R}^{\inf} \frac{kq_1q_2}{r^2} dr  \\
&=\frac{kQ_1Q_2}{R}
\end{align*}
\subsection{Positronium decay}
\begin{align*}
0=K_{E^-}+K_{E^+}-u_{electric}
\end{align*}
\subsubsection{Note}
\begin{itemize}
\item The total energy of any system is zero.
\item Newton's Law will break down when speed of a particle is not negligible compared to speed of light.
\end{itemize}

\subsection{Electrical Potential Energy of an Assembly of Charges} 
\begin{align*}
u_{elec}=\Sigma{\frac{qiqj}{r_{ij}}}
\end{align*}

\pagebreak
\section{Electric Potential}
How do we quantify the amount of charge distribution.\\
\begin{align*}
V_{P}=\frac{U_q+sources}{q}
\end{align*}
$JC^{-1} = V $= electrical potential energy per unit charge.

\subsection{A Point Charge}
\begin{align*}
V_{P}=\frac{kq}{r}
\end{align*}
If there is more than one charge just add them all up using scalar addition.
\end{document}

\subsection{

