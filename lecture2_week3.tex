\documentclass[12pt]{article}
\begin{document}

\title{Symmetry, Electric Flux and Gauss' Law}
\date{7 August, 2012}
\author{Gordon Ng}
\maketitle

\pagebreak
\tableofcontents
\pagebreak
\begin{itemize}
\section{Introduction to Symmetry}
\item It is fundamental to our daily lives.
\item An opreation under which a given object is unchanged.
\subsection{Example}
\item Rotating a circle about it's centre. It is also known as continous symmetry.
\item Reflecting a triangle. It is known as \textbf{discrete symmetry}. 
\section{Symmetry of Charge Distribution}
\item The symmetry of the electric field must match the symmetry of the charge distribution.
\item Symmetry in translation
\item If a infinitely long line is rotated at it's own axis it is unchanged.
\item Symmetry in rotation.
\item In \textbf{Cylindrical symmetry}, the charge must point away or towards the axis.
\item In \textbf{planar symmetry}, it has reflection , slide  and rotation symmetry, therefore the electrical field is perpendicular to the plane.
\item For a \textbf{Sphere}, if the sphere is displaced it is not the same sphere anymore.
\pagebreak
\section{Electric Flux}
\item Flux measures the amount of electric field passing through a surface of an area A when it is titled at angle $\theta$ from the field.
\begin{displaymath}
  \Phi_e = E_{\bot}A=EAcos \theta = \vec{E} \cdot  \vec{A}
\end{displaymath}
\subsection{Surface Intergrals}
\begin{displaymath}
\Phi_e= \sum  \vec{E} \cdot \delta  \vec{A} \longrightarrow \int \vec{E} \cdot d \vec{A}
\end{displaymath}
\item \textbf{Note:} If the electric field is everywhere \textbf{tangent} to the surface then $\phi_e =0$
\item \textbf{Note:} If the electric is \textbf{perpendicular} everywhere to the surface then $\phi_e=EA$
\subsection{closed surface}
\item A closed surface has a distinct inside and outside. By definition the vector dA points towards outside.
\begin{displaymath}
\Phi_e= \sum  \vec{E} \cdot \delta  \vec{A} \longrightarrow  \vec{E} \cdot d \vec{A}
\end{displaymath}
\item Flux flowing through the same circular area has same flux.

\pagebreak
\section{Gauss's Law}
\item no enclosed charges = no net electric flux
\item It is also applicable to electrodynamics.
\item
\begin{displaymath}
\phi_e= \int \vec{E} \cdot d \vec{A} = \frac{Q_{enclosed}}{\varepsilon_0}
\end{displaymath}
\section{Conductors in electrostatic Equiribrium}
\item Electric Field inside a conducter is zero. 
\item All excess charge is on he surface.  
\item The electric field at the surface is perpendicular to the surface.
\item The electric field is perpendicular to all conducting surfaces.
\subsection{Shielding}
\item If there is a hollow hole (Gaussian Surface) enclosed by a conductor, the electric flux inside this hole is zero. 
\item If a positive charge is put into the hole. The electric field of the conductor is also zero. THe inner surface has charge of -q and outer surface has +q.
\item Vice Versa.
\end{itemize}
\end{document}