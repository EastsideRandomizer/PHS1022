\documentclass[12pt]{article}
\usepackage{amsmath}
\usepackage{tikz}
\begin{document}

\title{Change in Voltage}
\date{17 August, 2012}
\author{Gordon Ng}
\maketitle
\pagebreak
\tableofcontents
\pagebreak
\section{Limit of Classical Physics}
Let's say:
\begin{align}
E= \frac{\varepsilon_O A}{d}
\end{align}
if $d \gg 10^{-10} \angstrom$ the Classical law holds. But anywhere close than that we must use relativistic mechanics. 

\section{Relativistic mechanics} 
Relativistic mechanics encloses newtonian mechanics.
\subsection{Examples}
\begin{align}
E^2=(pc)^2+(mc^2)^2
\end{align}
If:
\begin{align}
p &= 0, E=mc^2 \\
m &= 0, E=pc=\frac{hc}{\lambda}=hf \\
V \ll c &= mv \ll mc = pc \ll mc^2 \\
\text{Hence:} \\
E&=\sqrt{p^2c^2+m^2c^4}=mc^2+\frac{p^2}{2m}
\end{align}
\subsection{Two Lines of infinite charge}
\begin{align}
|E|=\frac{\eta}{\varepsilon_O}
\end{align}
If E is very large Relativistic effects will allow a electron-positron pairs to be produced. Then is causes polarisation in the electric field, acting like a dielectric material.

\subsubsection{Dielectric Constant when E is very large}
\begin{align}
K&=\frac{E_O}{E}=\frac{E_0}{E_o-E_{induced}} \\
K_{vac}&=\frac{E_0}{E_O-N\frac{\frac{1}{2}\varepsilon_OE_0^2}{2mec^2}} \\
K_{vac}&=\frac{1}{1-N\frac{\frac{1}{2}\varepsilon_OE_0}{4mec^2}}
\end{align}
$K \to \infty$  as $E_O \to \infty$

\section{Point Charge}
Classical model of Force between two charges:
\begin{align}
F=\frac{KQ_1Q_2}{R^2}
\end{align}
As $R \to 0$, F does not tend to $\infty$ due to dressed electron effect. \\
Hence Compton Effect comes into play:
\begin{align}
\end{align}
\section{Link Between Physics}
  \def\firstcircle{(90:1.75cm) circle (2.5cm)}
  \def\secondcircle{(210:1.75cm) circle (2.5cm)}
  \def\thirdcircle{(330:1.75cm) circle (2.5cm)}
    \begin{tikzpicture}
      \begin{scope}
    \clip \secondcircle;
    \fill[cyan] \thirdcircle;
      \end{scope}
      \begin{scope}
    \clip \firstcircle;
    \fill[cyan] \thirdcircle;
      \end{scope}
      \draw \firstcircle node[text=black,above] {Electro Magnetism };
      \draw \secondcircle node [text=black,below left] {Rotational Mechanics};
      \draw \thirdcircle node [text=black,below right] {Quantam mechanics };
    \end{tikzpicture}
\end{document}
