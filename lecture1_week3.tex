\documentclass[12pt]{article}
\begin{document}
\section{Electric Field}
\begin{displaymath}
E=\frac{kQ}{r^2}\vec{r}
\end{displaymath}
\begin{displaymath}
E(x)=\frac{k(+Q)}{(x-\frac{1}{2}s)^2}+\frac{k(-Q)}{(x+\frac{1}{2}s)^2}=\frac{2ksqx}{(x+\frac{1}{2}s)^2(x-\frac{1}{2}s)^2}
\end{displaymath}
\subsection{dipole moment}
The dipole moment $\vec{p}$ is a vector pointing from the negative to the positive dipole with a magnitude of $qs$
\subsection{Pictorial Representation of Electric Field}
The same situation can be depicted  Field-vector diagram and Field-line diagram.
\subsubsection{Field-vector diagram}
\subsubsection{Field-line diagram}
At any point on a field line diagram, the vector which represents field strength is the tangent vector of a field line at any point. \\
\begin{itemize}
\item They do not formed closed loops.
\item Field lines cannot simultaneously emerge and converge at one point.
\item There must be a source of field lines.
\end{itemize}
\subsubsection{Symmetry}
It has a important role in physics.

\subsection{Continous Charge Distribution}
The Linear charge density $\lambda$ of an object of length L and charge Q is defined as: \\
\begin{displaymath}
\lambda=\frac{Q}{L}
\end{displaymath}
\subsection{Electric Field Strength}
\begin{itemize}
\item Point Charge:
\begin{displaymath}
\frac{1}{4\pi\varepsilon_0}\frac{q}{r^2}
\end{displaymath}
\end{itemize}
\subsection{On-axis Electric FIeld}
\begin{itemize}
\item Choose a coordinate system
\item Divide the ring into segments
\item Note Symmetry or use calculus
\end{itemize}
Z component at point P:
\begin{displaymath}
(E_i)_z = E_i cos \theta_i = \frac {k \delta q cos \theta}{r^2}
\end{displaymath}
\begin{displaymath}
(E_i)_z = E_i cos \theta_i = \frac {k \delta q Z }{(r^2+z^2)^{\frac{3}{2}}}
\end{displaymath}

\subsubsection{Disk}
A disk consists infinite number of hoops.

\end{document}
