\documentclass[12pt]{article}
\usepackage{amsmath}
\begin{document}

\title{Energy of Electric Charges}
\date{15 August, 2012}
\author{Gordon Ng}
\maketitle

\pagebreak
\tableofcontents
\pagebreak
\section{Potential Energy of a dipole}
\begin{align}\
\Delta U_{dipole} &= - pE cos \Phi \\
&= -\vec{p} \cdot E \\ \\
p = qE
\end{align}
The electric field does work on the particle and gives it kinetic energy. This decreases the electric field's energy. 
\begin{itemize}
\item at $\Phi = 0^{\circ}$, the dipole is in a stable equilibrium.
\end{itemize}
\section{Electric Potential Energy}
$U_{elec}=U_o+qEs$
It is an analogue of gravitational Potential Energy.
\textbf{Note:} s is the distance from the negative terminal.
\subsection{Electric Potential Inside a Parallel-Plate Capacitor}
\begin{align}
V=Es \\
E = \frac{\Delta V_c}{d} \\
\text{Dimensions: }
[E]= \frac{N}{C}\frac{V}{m}
\end{align}
The is transform one description of electric field to another.
\begin{itemize}
\item s is once again the distance from the negative terminal.
\item The energy gained inside a electric field of length L is is the same.
\end{itemize}
\subsubsection{Example}
$Q= \pi R^2 \varepsilon_0 E$ \\
For any capacitor: \\
$Q=\text{Area $\times$ Charge per Unit Area}$
\pagebreak
\section{Electric Potential of A Charged Sphere}
$Q=4 \pi \varepsilon_0 R V_0$
\section{Charge}
\subsection{Ring of Charge}
\begin{align*}
V_p(z)=\frac{kq}{\sqrt{R^2+z^2}}
\end{align*}
\subsection{Disk of Charge}
\begin{align*}
V_p=\frac{Q}{2 \pi \varepsilon_0 R^2}(\sqrt{R^2+Z^2}-|Z|)
\end{align*}
\end{document}
