\documentclass[12pt]{article}
\begin{document}
\section{Question 1}
An object which is treated as point mass rotating in a circle. \\
The object is connected to the axis of rotation by a rope. \\
Initally it is 0.22m long and the object is rotating at $0.55 \ rads^{-1}$ \\
The rope is \textbf{shortened} to 0.15m and we are required to find the final angular speed.\\

First some variables are declared: \\ 
\begin{list*}
$v_i =$ Inital linear speed of the object $=unknown$ \\
$\omega_i =$  Unital angular speed of the object. $= 0.55 \ rads^{-1} $ \\
$r_i =$ Unital length of the rope. $= 0.22m $ \\
$v_f =$ Final linear speed of the object $= unknown$ \\
$\omega_f =$  Final angular speed of the object. $ = unknown$\\
$r_f =$ Final length of the rope. $ = 0.15m$
\end{list*}
\begin{itemize}
\item We start with finding the inital linear speed of the ball: \\
The inital linear speed can be found by using $v_i=\omega_ir_i$. \\
The values of $\omega_i$ and $r_i$ are substitued in and $v_i = 0.121 \ ms^{-1}$
\item As no net torque acts on the system, angular momentum is conserved. Hence the formulae $mr_iv_isin\beta=mr_fv_fsin\beta$ can be used to find the final linear velocity and hence angular speed) of the object: \\
\begin{displaymath}
mr_iv_isin\beta&=mr_fv_fsin\beta \\
\end{displaymath}
For a circular orbit $sin\beta=1$ and mass cancels out: \\
\begin{displaymath}
r_iv_i=r_fv_f \\
\end{displaymath}
But $v_f=\omega_fr_f$: \\
\begin{displaymath}
r_iv_i=r_f^2\omega_f \\
\end{displaymath}
Hence:
\begin{displaymath}
\omega_f &= \frac{r_iv_i}{r_f^2}=\frac{0.22\times0.121}{0.15^2}=1.18 \ rads^{-1}
\end{displaymath}
\textbf{Q.E.D}
\end{list*}
\pagebreak
\section{Question 2}
In question 2 we assume $\vec{a},\vec{b},\vec{c}$ are vectors of non-zero length. \\
Put: \\
\begin{equation}
\vec{a}=ai+bj+ck\\
\end{equation}
\begin{equation}
\vec{b}=di+ej+fk\\
\end{equation}
\begin{equation}
\vec{c}=xi+yj+zk\\
\end{equation}
We start with LHS: \\
\begin{displaymath}
(\vec{a} \times \vec{b}) \times \vec{c}
\end{displaymath}
\begin{displaymath}
\vec{a} \times \vec{b}=(bf-ce)i-(af-cd)j+(ae-bd)k
\end{displaymath}
\begin{displaymath}
\vec{a} \times \vec{b}=(bf-ce)i+(cd-af)j+(ae-bd)k
\end{displaymath}

\begin{equation}
\tilde{a}=2i+3j+6k \\
\end{equation}
\begin{equation}
\tilde{b}=3i+5j+7k \\
\end{equation}
\begin{equation}
\tilde{c}=4i+2j+k \\
\end{equation}
\begin{equation}


\pagebreak
\section{Question 3}
In question 3 we treat the satelite as a point mass.
\subsection{Part A}
There is no net torque on the satelite because the only force(gravitational force) is acting in the same direction as the moment arm (along the radius between the satelite and the planet). 
\subsection{Part B}
Because there is no net torque acting on the system, the formulae of conservation of momentum can be used: \\
\begin{displaymath}
mr_av_asin\beta=mr_bv_bsin\beta
\end{displaymath}
The term m(mass of the satelite) cancels out. \\
At both point a and b the linear momentum vector $\vec{p}$ is perpendicular to $\vec{r}$ hence at both points $sin\beta=1$: \\
\begin{displaymath}
r_av_a=r_bv_b
\end{displaymath}
Rearriangement results in: \\
\begin{displaymath}
\frac{r_av_a}{r_b}=v_b
\end{displaymath}
$r_a$ is the distance between the centre of planet and point a which is: \\
$30000-9000-15000=6000 \ km$  \\
$r_b$ is distnce from centre of planet and point b which is : \\ $15000+9000 = 24000 \ km$ \\
Both Figure can be left in km as it cancels out. \\
In the end:
\begin{displaymath}
v_b=\frac{r_av_a}{r_b}=\frac{6000\times8000}{24000}=2000ms^{-1}
\end{displaymath}
\pagebreak
\section{Question 4}
\begin{itemize}
\item Set up an inclined plane at at small angle (maxmium $20 \cdot$ to the horizontal)
\item Rule out a line across the plane which is used as the starting point.
\item At some distance down the plane rule out another line which is used as end point.
\item Place each sphere on the start point and let it roll down the hill.
\item Measure the time required for each sphere to travel from start to end point using a stopwatch.
\item Repeat the experiment for each sphere for three times and calculate the average time required.
\item The sphere which takes longer to roll between two point is the hollow sphere.
\end{itemize}
\end{document}