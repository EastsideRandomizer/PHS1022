
\documentclass[10pt]{article}
\usepackage{amsmath}

\begin{document}
\section{Question 1}
An object which is treated as point mass rotating in a circle. \\
The object is connected to the axis of rotation by a rope. \\
Initally it is 0.22m long and the object is rotating at $0.55 \ rads^{-1}$ \\
The rope is \textbf{shortened} to 0.15m and we are required to find the final angular speed.\\

First some variables are declared: \\ 
\begin{itemize}
\item $v_i =$ Inital linear speed of the object $=unknown$ \\
$\omega_i =$  Unital angular speed of the object. $= 0.55 \ rads^{-1} $ \\
$r_i =$ Unital length of the rope. $= 0.22m $ \\
$v_f =$ Final linear speed of the object $= unknown$ \\
$\omega_f =$  Final angular speed of the object. $ = unknown$\\
$r_f =$ Final length of the rope. $ = 0.15m$

\item We start with finding the inital linear speed of the ball: \\
The inital linear speed can be found by using $v_i=\omega_ir_i$. \\
The values of $\omega_i$ and $r_i$ are substitued in and $v_i = 0.121 \ ms^{-1}$
\item As no net torque acts on the system, angular momentum is conserved. Hence the formulae $mr_iv_isin\beta=mr_fv_fsin\beta$ can be used to find the final linear velocity and hence angular speed) of the object: \\
\begin{displaymath}
mr_iv_isin\beta=mr_fv_fsin\beta \\
\end{displaymath}
For a circular orbit $sin\beta=1$ and mass cancels out: \\
\begin{displaymath}
r_iv_i=r_fv_f \\
\end{displaymath}
But $v_f=\omega_fr_f$: \\
\begin{displaymath}
r_iv_i=r_f^2\omega_f \\
\end{displaymath}
Hence:
\begin{displaymath}
\omega_f = \frac{r_iv_i}{r_f^2}=\frac{0.22\times0.121}{0.15^2}=1.18 \ rads^{-1}
\end{displaymath}
\textbf{Q.E.D}
\end{itemize}
\pagebreak
\section{Question 2}
We are told to show that $ (\vec{a} \times \vec{b}) \times \vec{c}$ \textbf{does not always equal to} $\vec{a} \times (\vec{b} \times \vec{c})$ \\ 
In question 2 we assume $\vec{a},\vec{b},\vec{c}$ are vectors of non-zero length. \\
Put: \\
\begin{equation}
\vec{a}=a_xi+a_yj+a_zk\\
\end{equation}
\begin{equation}
\vec{b}=b_xi+b_yj+b_zk\\
\end{equation}
\begin{equation}
\vec{c}=c_xi+c_yj+c_zk\\
\end{equation}
We start with $(\vec{a} \times \vec{b}) \times \vec{c}$ :
\begin{align*}
\vec{a} \times \vec{b}&=(a_yb_z-a_zb_y)i-(a_xb_z-a_zb_x)j+(a_xb_y-a_yb_x)k \\
&=(a_yb_z-a_zb_y)i+(a_zb_x-a_xb_z)j+(a_xb_y-a_yb_x)k
\end{align*}

Hence:
\begin{align*}
\left( \vec{a} \times \vec{b} \right) \times \vec {c} &=
\begin{vmatrix}
  \vec{i} & \vec{j}& \vec{k}\\
  a_yb_z-a_zb_y & a_zb_x-a_xb_z & a_xb_y-a_yb_x \\
  c_x &c_y &c_z \\
\end{vmatrix} \\
&=[(a_zb_x-a_xb_z)c_z-(a_xb_y-a_yb_x)c_y]\vec{i} \\
&-[(a_yb_z-a_zb_y)c_z-(a_xb_y-a_yb_x)c_x]\vec{j} \\ 
&+ [(a_yb_z-a_zb_y)c_y-(a_zb_x-a_xb_z)c_x]\vec{j} \\ 
&\text{Break this seemingly long vector down to i, j and k components} \\
&\text{We only look at the i component:} \\
&i=[(a_zb_x-a_xb_z)c_z-(a_xb_y-a_yb_x)c_y]\vec{i} \\
&i=a_zb_xc_z-a_xb_zc_z-a_xb_yc_y+a_yb_xc_y \\
&\text{If we add and subtract $a_xb_xc_x$ from the i component we get: } \\
&i=a_zb_xc_z-a_xb_zc_z-a_xb_yc_y+a_yb_xc_y+a_xb_xc_x-a_xb_xc_x \\ 
&\text{Collect like terms:} \\
&i=b_x(a_xc_x+a_zc_z+a_yc_y)-a_x(b_xc_x+b_yb_z+b_zc_z)\\
&\text{Note: $a_xc_x+a_zc_z+a_yc_y=a \cdot c$} \\
&\text{This means i component of $\left( \vec{a} \times \vec{b} \right) \times \vec {c}$ is }: \\
&i=(a \cdot c)b_x-(b \cdot c)a_x \\
&\text{Following the same method for j and k component we get :} \\
 \left( \vec{a} \times \vec{b} \right) \times \vec {c} &=[(a \cdot c)b_x-(b \cdot c)a_x] \vec{i}+[(a \cdot c)b_y-(b \cdot c)a_y]\vec{j}+[(a \cdot c)b_z-(b \cdot c)a_z]\vec{k} \\
&\text{Therefore: } \\ 
 &= (\vec{a} \cdot \vec{c})\vec{b}-(\vec{b} \cdot \vec{c})\vec{a}
\end{align*}

\pagebreak 

We go on with  with $(\vec{a} \times \vec{b}) \times \vec{c}$ :
\begin{align*}
\vec{b} \times \vec{c}&=(b_yc_z-b_zc_y)i-(b_xc_z-b_zc_x)j+(b_xc_y-b_yc_x)k \\
&=(b_yc_z-b_zc_y)i+(b_zc_x-b_xc_z)j+(b_xc_y-b_yc_x)k
\end{align*}
So:
\begin{align*} \vec{a} \times \left( \vec{b} \times \vec {c} \right) &=
\begin{vmatrix}
  \vec{i} & \vec{j}& \vec{k}\\
  a_x & a_y & a_z \\
  b_yc_z-b_zc_y & b_zc_x-b_xc_z & b_xc_y-b_yc_x \\
\end{vmatrix} \\
&=[a_y(b_xc_y-b_yc_x)-a_z(b_zc_x-b_xc_z)]\vec{i} \\
&-[a_x(b_xc_y-b_yc_x)-a_z(b_yc_z-b_zc_y)]\vec{j} \\ 
&+ [a_x(b_zc_x-b_xc_z)-a_y(b_yc_z-b_zc_y)]\vec{j} \\ 
&\text{Break this seemingly long vector down to i, j and k components} \\
&\text{We only look at the i component:} \\
i&=[a_y(b_xc_y-b_yc_x)-a_z(b_zc_x-b_xc_z)]\vec{i}\\
&=a_yb_xc_y-a_yb_yc_x-a_zb_zc_x+a_zb_xc_z \\
&\text{If we add and subtract $a_xb_xc_x$ from the i component we get: } \\
&=a_yb_xc_y-a_yb_yc_x-a_zb_zc_x+a_zb_xc_z+a_xb_xc_x-a_xb_xc_x \\
&\text{Collect like terms:} \\
&=b_x(a_xc_x+a_yc_y+a_zc_z)-c_x(a_xb_x+a_yb_z+a_zb_z) \\
&\text{Note: $a_xc_x+a_zc_z+a_yc_y=a \cdot c$} \\
&\text{This means i component of $ \vec{a} \times \left( \vec{b}  \times \vec {c} \right) $ is }: \\
&i=(a \cdot c)b_x-(a \cdot b)c_x \\
&\text{Following the same method for j and k component we get :} \\
  \vec{a} \times \left( \vec{b}  \times \vec {c} \right) &=[(a \cdot c)b_x-(a \cdot b)c_x] \vec{i}+[(a \cdot c)b_y-(a \cdot b)c_y]\vec{j}+[(a \cdot c)b_z-(a \cdot b)c_z]\vec{k}
\end{align*}
Hence:
\begin{displaymath}
  \vec{a} \times \left( \vec{b}  \times \vec {c} \right)  =(a \cdot c)b-(a \cdot b)c
\end{displaymath}
and
\begin{displaymath}
  \left( \vec{a} \times \vec{b} \right) \times \vec {c} =(a \cdot c)b-(b \cdot c)a
\end{displaymath}
\begin{itemize}
\item For \textbf{most} cases $(a \cdot c)b-(a \cdot b)c \neq (a \cdot c)b-(b \cdot c)a$ 
\item Hence: $\left( \vec{a} \times \vec{b} \right) \times \vec {c}$ \textbf{does not always equal to} $\vec{a} \times \left( \vec{b}  \times \vec {c} \right)$
\item \textbf{Note:} Add and subtract $a_yb_yc_y$ or $a_zb_zc_z$ instead of $a_xb_xc_x$ when doing j and k component.
\end{itemize}
\pagebreak
\subsection{Example}
We will show an example:
\begin{equation}
\tilde{a}=2i+3j+6k \\
\end{equation}
\begin{equation}
\tilde{b}=3i+5j+7k \\
\end{equation}
\begin{equation}
\tilde{c}=4i+2j+k \\
\end{equation}
So:
\begin{displaymath}
  \vec{a} \times \left( \vec{b}  \times \vec {c} \right)  =(a \cdot c)b-(a \cdot b)c
\end{displaymath}
\begin{displaymath}
  \left( \vec{a} \times \vec{b} \right) \times \vec {c} =(a \cdot c)b-(b \cdot c)a
\end{displaymath}

\begin{align*}
  \vec{a} \times \left( \vec{b}  \times \vec {c} \right)  &=(8+6+6)(3i+5j+7k)-(6+15+42)(4i+2j+k) \\
&= 60i+100j+140k-252i-126j-63k \\
&=-192i-26j+77k \\
 \left( \vec{a} \times \vec{b} \right) \times \vec {c} &= (8+6+6)(3i+5j+7k)-(12+10+7)(2i+3j+6k)\\
&= 60i+100j+140k-58i-87j-174k \\
&=2i+13j-34k
\end{align*} \\
\textbf{Conclusion:} For this case $\left( \vec{a} \times \vec{b} \right) \times \vec {c}$ \textbf{does not equal to} $\vec{a} \times \left( \vec{b}  \times \vec {c} \right)$
\pagebreak
\section{Question 3}
In question 3 we treat the satelite as a point mass.
\subsection{Part A}
There is no net torque on the satelite because the only force(gravitational force) is acting in the same direction as the moment arm (along the radius between the satelite and the planet). 
\subsection{Part B}
Because there is no net torque acting on the system, the formulae of conservation of momentum can be used: \\
\begin{displaymath}
mr_av_asin\beta=mr_bv_bsin\beta
\end{displaymath}
The term m(mass of the satelite) cancels out. \\
At both point a and b the linear momentum vector $\vec{p}$ is perpendicular to $\vec{r}$ hence at both points $sin\beta=1$: \\
\begin{displaymath}
r_av_a=r_bv_b
\end{displaymath}
Rearriangement results in: \\
\begin{displaymath}
\frac{r_av_a}{r_b}=v_b
\end{displaymath}
$r_a$ is the distance between the centre of planet and point a which is: \\
$30000-9000-15000=6000 \ km$  \\
$r_b$ is distnce from centre of planet and point b which is : \\ $15000+9000 = 24000 \ km$ \\
Both Figure can be left in km as it cancels out. \\
In the end:
\begin{displaymath}
v_b=\frac{r_av_a}{r_b}=\frac{6000\times8000}{24000}=2000ms^{-1}
\end{displaymath}
\pagebreak
\section{Question 4}
\begin{itemize}
\item Set up an inclined plane at at small angle (maxmium $20 \cdot$ to the horizontal)
\item Rule out a line across the plane which is used as the starting point.
\item At some distance down the plane rule out another line which is used as end point.
\item Place each sphere on the start point and let it roll down the hill.
\item Measure the time required for each sphere to travel from start to end point using a stopwatch.
\item Repeat the experiment for each sphere for three times and calculate the average time required.
\item The sphere which takes longer to roll between two point is the hollow sphere.
\end{itemize}

\section{Question 5}
As:
\begin{displaymath}
1 e^{-} = 1.602 \times 10^{-19} C
\end{displaymath}
Therefore:
\begin{displaymath}
1 C = \frac{1}{1.602 \times 10^{-19} \frac{C}{e^{-}}}= 6.24 \times 10^{18} electrons
\end{displaymath}
One simply(which is an understatement in its own right) has to remove $6.24 \times 10^{18}$ electrons to give a blade of grass 1 Coulomb of charge. 

\pagebreak
\section{Question 6}
In this case a diagram is very useful.
\\ \\ \\ \\ \\ \\ \\ \\ \\ \\ \\ \\ \\ \\ \\ \\
Magnitude of Force q1 acting on q3:
\begin{align*}
|F_{q1 on q3}|&=\frac{kq_1q_3}{r^2} \\
&=\frac{9.0\times 10^9 \times 5 \times 10^{-8} \times 3\times 10^{-8}}{(0.1)^2} \\
&=1.35 \times 10^{-3} N
\end{align*}
As opposite charges attract, $q_3$ experiences a downward force towards $q_1$ \\
Hence $F_{q1 on q3}=-1.25 \times 10^{-3} N \vec{j} $ \\

Magnitude of Force q2 acting on q3:
\begin{align*}
|F_{q2q3}|&=\frac{kq_2q_3}{r^2} \\
&=\frac{9.0\times 10^9 \times 3 \times 10^{-8} \times 3\times 10^{-8}}{(0.1)^2+(0.05)^2} \\
&=6.48 \times 10^{-4} N
\end{align*}
Force q2 acting on q3 can be spilt into i and j component
\begin{align*}
F_{q2q3}&=|F_{q2 on q3}| cos \Phi \vec{i}+ |F_{q2 on q3}| sin \Phi \vec{j} \\
&=- 6.48 \times 10^{-4} \frac{5}{\sqrt{125}} \vec{i}+ 6.48 \times 10^{-4}  \frac{10}{\sqrt{125}} \vec{j}
\end{align*} \\
Add up the components of the forces together:
\begin{align*}
F_{q3}&= 6.48 \times 10^{-4} \frac{5}{\sqrt{125}} \vec{i}+ 6.48 \times 10^{-4}  \frac{10}{\sqrt{125}} \vec{j} - 1.35 \times 10^{-3} j \\
&=-2.898\times 10^{-4} \vec{i}+ 7.704 \times 10^{-4} \vec{j}
\end{align*}
The magnitude of Force on charge 3 is:
\begin{align*}
|F_q3|&=\sqrt{(-2.898\times 10^{-4})^2+ (7.704 \times 10^{-4})^2} \\
&=8.23\times 10^{-4} N
\end{align*}
The angle is:
\begin{align*}
\theta &= tan^{-1} \left|\frac{F_j}{F_i} \right|
        =tan^{-1} \left(\frac{7.704 \times 10^{-4}}{2.898\times 10^{-4}} \right) \\
       &=69.38^{\circ} \approx 69.4^{\circ}
\end{align*}

\section{Question 7}
Each $Na^+$ and $Cl^{-}$ ion has either positive charge or negative charge of 1, meaning they both have
$1.602\times10^{-19}$ Coulombs of charge. \\ 
Internucleus distance D is $2.82 \times 10^{10}$m. \\
The magnitude of the electric force of attraction is simply:
\begin{displaymath}
|F_{attraction}|=\frac{k|q_{Na^+}||q_{Cl^-}|}{D^2}
\end{displaymath}
Throw the numbers in:
\begin{displaymath}
|F_{attraction}|=\frac{9.0 \times 10^9 \times(1.6 \times 10^{-19})^2 }{(2.82 \times 10^{-10})^2}=2.90\times 10^{-9} \ N 
\end{displaymath}

\pagebreak
\section{Question 8}
\subsection{Part A}
We are given the function which describes the strength of magnetic field along the z-axis:
\begin{displaymath}
E(z)=\frac{Q}{4 \pi \varepsilon_0}\frac{z}{(z^2+R^2)^{\frac{3}{2}}}
\end{displaymath}
\begin{itemize}
\item We are told to derive the formulae for the z-coordinate where the electrical field is at a maximum.
\item The term ``maximum'' means differentiation is our friend here:
\setcounter{equation}{0}
\begin{align*}
E'(z)&=\frac{Q}{4 \pi \varepsilon_0} \frac{ d \left(\frac{z}{(z^2+R^2)^{\frac{3}{2}}} \right) }{dz} \\
&=\frac{Q}{4 \pi \varepsilon_0} \left(z \times  \frac{d(z^2+R^2)^{-\frac{3}{2}}}{dz} + \frac{dz}{dz} \times (z^2+R^2)^{-\frac{3}{2}}\right) \\
&=\frac{Q}{4 \pi \varepsilon_0} \left(- 3 z^2 \times (z^2+R^2)^{-\frac{5}{2}} + (z^2+R^2)^{-\frac{3}{2}}\right)
\end{align*}

 Because we are trying to find maximum let $E'(z)=0$:
\begin{align*}
0&=\frac{Q}{4 \pi \varepsilon_0} \left(- 3 z^2 \times (z^2+R^2)^{-\frac{5}{2}} + (z^2+R^2)^{-\frac{3}{2}}\right) \\
&=- 3 z^2 \times (z^2+R^2)^{-\frac{5}{2}} + (z^2+R^2)^{-\frac{3}{2}} \\
3 z^2 \times (z^2+R^2)^{-\frac{5}{2}} &= (z^2+R^2)^{-\frac{3}{2}} \\
\frac{3 z^2}{ z^2+R^2} &= 1 \\
z^2 &= \frac{R^2}{2} \\
z^2 &= \pm\frac{R}{\sqrt{2}} \ But \ z= \frac{R}{\sqrt{2}} \ as \ z > 0
\end{align*}
\item The formula for the location of maximum electric field $z= \frac{R}{\sqrt{2}}$.
\pagebreak
\subsection{Part B}
\item We start this part by finding the electric field at the maximum point:
\begin{align*}
E(z)&=\frac{Q}{4 \pi \varepsilon_0}\frac{z}{(z^2+R^2)^{\frac{3}{2}}} \\
E \left ( \frac{R}{\sqrt{2}} \right )&=\frac{kQ  \frac{R}{\sqrt{2}}}{ \left ( \left(\frac{R}{\sqrt{2}} \right)^2+R^2 \right)^{\frac{3}{2}}} \\
E \left ( \frac{R}{\sqrt{2}} \right )&=\frac{kQ  \frac{R}{\sqrt{2}}}{ \left( \frac{3R^2}{2} \right)^{\frac{3}{2}}} \\
\end{align*}
\item It isn't a easy task to simplfly the answer so it will be left in this form.
\item We do have the formula for finding the force acting on a charge as a result of the electric field:
\begin{displaymath}
F_{Electric}=\vec{q}\vec{E}
\end{displaymath}
\item Hence:
\begin{align*}
F_{Electric}&=\vec{Q}\times \frac{kQ  \frac{R}{\sqrt{2}}}{ \left( \frac{3R^2}{2} \right)^{\frac{3}{2}}} \\
&= \frac{kQ^2  \frac{R}{\sqrt{2}}}{ \left( \frac{3R^2}{2} \right)^{\frac{3}{2}}} \\
\end{align*}
\pagebreak
\subsubsection{The Maximum Mass of The ball}
\item The maximum mass is when the ball is in static equilibrium. (Net Force on ball = 0)
\item Because the ring is oriented horizontally, Gravitational Force acts on the ball.
\item Both the ring and the ball has a positive charge so they repel each other.
\item We start with drawing the free-body diagram of the ball:
\\ \\ \\ \\ \\ \\ \\ \\ \\ \\ \\ \\ \\ \\ \\
\item Equations of motion: 
\begin{align*}
F_{electric}-Mg&=0 \\
Hence:
\frac{kQ^2  \frac{R}{\sqrt{2}}}{ \left( \frac{3R^2}{2} \right)^{\frac{3}{2}}} - Mg &= 0 \\
M &= \frac{kQ^2  \frac{R}{\sqrt{2}}}{ \left( \frac{3R^2}{2} \right)^{\frac{3}{2}}g} = \frac{kQ^2}{ \sqrt{2} \left(\frac{3}{2} \right)^{\frac{3}{2}}g R^2}
\end{align*}
\item This seems messy but at least it doesn't dependent on how far along the z axis is. As expected the maximum mass decreases as radius of the ring \textbf{increases}.
\end{itemize}
\end{document}